\documentclass{article}

% if you need to pass options to natbib, use, e.g.:
% \PassOptionsToPackage{numbers, compress}{natbib}
% before loading nips_2017
%
% to avoid loading the natbib package, add option nonatbib:
% \usepackage[nonatbib]{nips_2017}

\usepackage[final]{ner}

% to compile a camera-ready version, add the [final] option, e.g.:
% \usepackage[final]{nips_2017}

\usepackage[utf8]{inputenc} % allow utf-8 input
\usepackage[T1]{fontenc}    % use 8-bit T1 fonts
\usepackage{hyperref}       % hyperlinks
\usepackage{url}            % simple URL typesetting
\usepackage{booktabs}       % professional-quality tables
\usepackage{amsfonts}       % blackboard math symbols
\usepackage{nicefrac}       % compact symbols for 1/2, etc.
\usepackage{microtype}      % microtypography

%% abbreviations
\newcommand{\ie}{\textit{i.e.}}
\newcommand{\st}{\textit{s.t.}}
\newcommand{\eg}{\textit{e.g.}}
\newcommand{\etc}{\textit{etc}}
\newcommand{\etal}{\textit{et~al.}}
\newcommand{\tuple}[1]{\ensuremath{( #1 )}\xspace}


\title{NER: a platform for robot communicating with other agents}

% The \author macro works with any number of authors. There are two
% commands used to separate the names and addresses of multiple
% authors: \And and \AND.
%
% Using \And between authors leaves it to LaTeX to determine where to
% break the lines. Using \AND forces a line break at that point. So,
% if LaTeX puts 3 of 4 authors names on the first line, and the last
% on the second line, try using \AND instead of \And before the third
% author name.

% An example of author info
%Chen Min\thanks{Use footnote for providing further
%information about author (webpage, alternative
%address)---\emph{not} for acknowledging funding agencies.} \\
%Department of Computer Science\\
%NUS, Singapore\\
%\texttt{chenmin@comp.nus.edu.sg} \\

\author{
  Chen Min, PhD\thanks{Expected to graduate in Aug 2018}\\
  NEO Robotics\\
  NUS, Singapore\\
  \texttt{chenmin@comp.nus.edu.sg} \\
  \And
  Xie Shudong, PhD\thanks{Expected to graduate in Dec 2017}\\
  NEO Robotics\\
  Singapore\\
  \texttt{purplebamboot@gmail.com} \\
  \And
  Chen Liyuan, Msc\\
  NEO Robotics\\
  Singapore\\
  \texttt{colacly@gmail.com} \\
  %% examples of more authors
  %% \And
  %% Coauthor \\
  %% Affiliation \\
  %% Address \\
  %% \texttt{email} \\
  %% \AND
  %% Coauthor \\
  %% Affiliation \\
  %% Address \\
  %% \texttt{email} \\
  %% \And
  %% Coauthor \\
  %% Affiliation \\
  %% Address \\
  %% \texttt{email} \\
  %% \And
  %% Coauthor \\
  %% Affiliation \\
  %% Address \\
  %% \texttt{email} \\
}

\begin{document}
% \nipsfinalcopy is no longer used

\maketitle

\begin{abstract}
	
More and more robots are coming into human's daily life,
\eg, autonomous driving car and household robot.
Unlike traditional robots in the factory, robots that share the
same workspace with humans have to gather information on humans'
physical state and mental state.
Consider autonomous driving as an example, in order to be safe
and efficient, the robot car has to
know the positions of nearby human driven cars and the paths that 
human drivers are following.
Currently, robots mostly rely on onboard sensors to gather 
information, such as lidar and camera.
However, those sensors might fail due to weather or lighting
conditions, which might result in fatal accidents.
% topic
This paper introduces NEO-Robotics (NER), a platform built on NEO 
blockchain
for robot communicating with other agents, which includes humans and
other robots.
% the adavantage of NER, and the impact

The question remains is how robots communicate with each other and any other human-controlled robots peer to peer, without relying on a third-trust-party. This paper describes Neo-Robotics(NER), a new Application that enables robots to communicate with each other using NEO blockchain technology. NER can be installed on robots such as automatic cars and serving robots in hospitals and it will also be developed into a mobile App which is more user friendly. 
 
\end{abstract}


\section{Introduction}

Introduction goes here.



\section{Background}

\subsection{NEO blockchain}

Introduction of NEO blockchain.



\subsection{Existing problems with robotics communication}

This part may introduce the current problems faced by robotics in information sharing.



\section{NER}
What NER is, what NER can constibute to robotics communication.

\subsection{Motivation}
Why need this App.



\subsection{Blockchain based device}
Technology goes here: the basic ideas of this App and detail methodology used?



\subsubsection{Functionalities: PC}
What can this App do on PC.



\subsubsection{Functionalities: Mobile App}
What will the App do on mobile or other portable instruments.



\subsection{Implementation}
We can demostrate some interesting examples in this part.

\subsubsection{Case Study 1: Seeing through the walls}



\subsubsection{Case Study 2: Avoid the Tesla fatal crash while using autopilot mode}




\subsection{Storage of data shared}
Where and how we gonna store the data.



\subsection{Reward scheme of the App}
What is the reward people can gain by sharing the information.



\section{Current progress and time schedule}
Here we introduce our current progress and what is our plan for next step.

\subsection{Incorporation}


\subsection{Technology}

\subsection{Time schedule}
We are going to achieve what goals by what time.

%
%\paragraph{Paragraphs}
%
%There is also a \verb+\paragraph+ command available, which sets the
%heading in bold, flush left, and inline with the text, with the
%heading followed by 1\,em of space.
%
%\section{Citations, figures, tables, references}
%\label{others}
%
%These instructions apply to everyone.
%
%\subsection{Citations within the text}
%
%The \verb+natbib+ package will be loaded for you by default.
%Citations may be author/year or numeric, as long as you maintain
%internal consistency.  As to the format of the references themselves,
%any style is acceptable as long as it is used consistently.
%
%The documentation for \verb+natbib+ may be found at
%\begin{center}
%  \url{http://mirrors.ctan.org/macros/latex/contrib/natbib/natnotes.pdf}
%\end{center}
%Of note is the command \verb+\citet+, which produces citations
%appropriate for use in inline text.  For example,
%\begin{verbatim}
%   \citet{hasselmo} investigated\dots
%\end{verbatim}
%produces
%\begin{quote}
%  Hasselmo, et al.\ (1995) investigated\dots
%\end{quote}
%
%If you wish to load the \verb+natbib+ package with options, you may
%add the following before loading the \verb+nips_2017+ package:
%\begin{verbatim}
%   \PassOptionsToPackage{options}{natbib}
%\end{verbatim}
%
%If \verb+natbib+ clashes with another package you load, you can add
%the optional argument \verb+nonatbib+ when loading the style file:
%\begin{verbatim}
%   \usepackage[nonatbib]{nips_2017}
%\end{verbatim}
%
%As submission is double blind, refer to your own published work in the
%third person. That is, use ``In the previous work of Jones et
%al.\ [4],'' not ``In our previous work [4].'' If you cite your other
%papers that are not widely available (e.g., a journal paper under
%review), use anonymous author names in the citation, e.g., an author
%of the form ``A.\ Anonymous.''
%
%\subsection{Footnotes}
%
%Footnotes should be used sparingly.  If you do require a footnote,
%indicate footnotes with a number\footnote{Sample of the first
%  footnote.} in the text. Place the footnotes at the bottom of the
%page on which they appear.  Precede the footnote with a horizontal
%rule of 2~inches (12~picas).
%
%Note that footnotes are properly typeset \emph{after} punctuation
%marks.\footnote{As in this example.}
%
%\subsection{Figures}
%
%All artwork must be neat, clean, and legible. Lines should be dark
%enough for purposes of reproduction. The figure number and caption
%always appear after the figure. Place one line space before the figure
%caption and one line space after the figure. The figure caption should
%be lower case (except for first word and proper nouns); figures are
%numbered consecutively.
%
%You may use color figures.  However, it is best for the figure
%captions and the paper body to be legible if the paper is printed in
%either black/white or in color.
%\begin{figure}[h]
%  \centering
%  \fbox{\rule[-.5cm]{0cm}{4cm} \rule[-.5cm]{4cm}{0cm}}
%  \caption{Sample figure caption.}
%\end{figure}
%
%\subsection{Tables}
%
%All tables must be centered, neat, clean and legible.  The table
%number and title always appear before the table.  See
%Table~\ref{sample-table}.
%
%Place one line space before the table title, one line space after the
%table title, and one line space after the table. The table title must
%be lower case (except for first word and proper nouns); tables are
%numbered consecutively.
%
%Note that publication-quality tables \emph{do not contain vertical
%  rules.} We strongly suggest the use of the \verb+booktabs+ package,
%which allows for typesetting high-quality, professional tables:
%\begin{center}
%  \url{https://www.ctan.org/pkg/booktabs}
%\end{center}
%This package was used to typeset Table~\ref{sample-table}.
%
%\begin{table}[t]
%  \caption{Sample table title}
%  \label{sample-table}
%  \centering
%  \begin{tabular}{lll}
%    \toprule
%    \multicolumn{2}{c}{Part}                   \\
%    \cmidrule{1-2}
%    Name     & Description     & Size ($\mu$m) \\
%    \midrule
%    Dendrite & Input terminal  & $\sim$100     \\
%    Axon     & Output terminal & $\sim$10      \\
%    Soma     & Cell body       & up to $10^6$  \\
%    \bottomrule
%  \end{tabular}
%\end{table}
%
%\section{Final instructions}
%
%Do not change any aspects of the formatting parameters in the style
%files.  In particular, do not modify the width or length of the
%rectangle the text should fit into, and do not change font sizes
%(except perhaps in the \textbf{References} section; see below). Please
%note that pages should be numbered.
%
%\section{Preparing PDF files}
%
%Please prepare submission files with paper size ``US Letter,'' and
%not, for example, ``A4.''
%
%Fonts were the main cause of problems in the past years. Your PDF file
%must only contain Type 1 or Embedded TrueType fonts. Here are a few
%instructions to achieve this.
%
%\begin{itemize}
%
%\item You should directly generate PDF files using \verb+pdflatex+.
%
%\item You can check which fonts a PDF files uses.  In Acrobat Reader,
%  select the menu Files$>$Document Properties$>$Fonts and select Show
%  All Fonts. You can also use the program \verb+pdffonts+ which comes
%  with \verb+xpdf+ and is available out-of-the-box on most Linux
%  machines.
%
%\item The IEEE has recommendations for generating PDF files whose
%  fonts are also acceptable for NIPS. Please see
%  \url{http://www.emfield.org/icuwb2010/downloads/IEEE-PDF-SpecV32.pdf}
%
%\item \verb+xfig+ "patterned" shapes are implemented with bitmap
%  fonts.  Use "solid" shapes instead.
%
%\item The \verb+\bbold+ package almost always uses bitmap fonts.  You
%  should use the equivalent AMS Fonts:
%\begin{verbatim}
%   \usepackage{amsfonts}
%\end{verbatim}
%followed by, e.g., \verb+\mathbb{R}+, \verb+\mathbb{N}+, or
%\verb+\mathbb{C}+ for $\mathbb{R}$, $\mathbb{N}$ or $\mathbb{C}$.  You
%can also use the following workaround for reals, natural and complex:
%\begin{verbatim}
%   \newcommand{\RR}{I\!\!R} %real numbers
%   \newcommand{\Nat}{I\!\!N} %natural numbers
%   \newcommand{\CC}{I\!\!\!\!C} %complex numbers
%\end{verbatim}
%Note that \verb+amsfonts+ is automatically loaded by the
%\verb+amssymb+ package.
%
%\end{itemize}
%
%If your file contains type 3 fonts or non embedded TrueType fonts, we
%will ask you to fix it.
%
%\subsection{Margins in \LaTeX{}}
%
%Most of the margin problems come from figures positioned by hand using
%\verb+\special+ or other commands. We suggest using the command
%\verb+\includegraphics+ from the \verb+graphicx+ package. Always
%specify the figure width as a multiple of the line width as in the
%example below:
%\begin{verbatim}
%   \usepackage[pdftex]{graphicx} ...
%   \includegraphics[width=0.8\linewidth]{myfile.pdf}
%\end{verbatim}
%See Section 4.4 in the graphics bundle documentation
%(\url{http://mirrors.ctan.org/macros/latex/required/graphics/grfguide.pdf})
%
%A number of width problems arise when \LaTeX{} cannot properly
%hyphenate a line. Please give LaTeX hyphenation hints using the
%\verb+\-+ command when necessary.

\subsubsection*{Acknowledgments}

Do we need this acknowleadgment section?

\section*{References}

References follow the acknowledgments. Use unnumbered first-level
heading for the references. Any choice of citation style is acceptable
as long as you are consistent. It is permissible to reduce the font
size to \verb+small+ (9 point) when listing the references. {\bf
  Remember that you can go over 8 pages as long as the subsequent ones contain
  \emph{only} cited references.}

\medskip

\small

\bibliographystyle{named}
\bibliography{ner}

\end{document}
